\chapter{Иcпользование мастера начала работы для создания приложенияна основе файла Excel}

Целью работы является приобретение навыков создания приложений с использованием
мастера начала работы QlickView.

\myImage{Выбираем "Новый документ", чтобы приступить к работе.}{1}{1}
\myImage{Мастер помогает нам выполнить первоначальную конфигурацию, выбирем данные, которые хотим анализировать.}{2}{2}
\myImage{После импорта данных программа предлагает нам подредактировать название колонок данных.}{3}{3}
\myImage{Сохраним полученный документ. так как мы пользуемся бесплатной лицензией он будет привязан к текущему компьютеру.}{4}{4}
\myImage{Менеджер создания графиков. Построим столбчатю диаграмму.}{5}{5}
\myImage{Выбирем измирения, по которым хотим посмотреть диаграмму. Посмотрим отношение Покупатель/количество товара.}{6}{6}
\myImage{Выберием представление данных, с которым нам будет удобно работать}{7}{7}
\myImage{Построим также отношение Товары/Количество. На графиках наглядно видно, что PageWeavers
приобрел больше всего товаров, при этом основная масса продаж является поштучечной (товар в одном экземпляре, купили более 400 раз.)}{8}{8}
\myImage{Выберем самую большую колонку на диаграмме Товар/Количество.
Это автоматически создаст фильтр и отобразит все поштучные продажи товара AWC}{9}{9}

\clearpage
\myImage{Разнообразные виды диаграм для предоставления данных}{10}{10}
Популярные типы диаграм:
\begin{itemize}
    \item гистограммы;
    \item линейные графики;
    \item круговые диаграммы;
    \item точечные диаграммы;
    \item радар;
    \item сеточные диаграммы;
    \item блочные диаграммы;
    \item датчики;
    \item диаграммы Мекко.
\end{itemize}
