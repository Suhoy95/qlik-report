%% Преамбула TeX-файла

% 1. Стиль и язык
\documentclass[utf8x]{G7-32} % Стиль (по умолчанию будет 14pt)
\usepackage[T2A]{fontenc}
\usepackage[russian]{babel}
% Остальные стандартные настройки убраны в preamble.inc.tex.
\include{preamble.inc}

% Настройки листингов.
\include{listings.inc}

% Полезные макросы листингов.
% Любимые команды

\newtheorem{theorem}{Теорема}
\newtheorem{definition}{Определение}

\newcommand{\Code}[1]{\textbf{#1}}


\newcommand{\myImage}[3]{
\begin{figure}[!ht]
    \centering
    \includegraphics[width=0.9\textwidth]{figures/#2}
    \caption{#1}
    \label{#3}
\end{figure}
}


\begin{document}

\pagestyle{empty}
\begin{center}
    Министерство образования и науки Российской Федерации\\
    ФГАОУ ВПО  «УрФУ имени первого Президента России Б. Н. Ельцина»\\
    Институт радиоэлектроники и информационных технологий - РтФ\\
    Департамент информационных технологий и автоматики
    \par
    \vspace{4.5cm}
    \Large{
      Иcпользование мастера начала работы Qlickview для создания приложенияна основе файла Excel


      \par
      \vspace{0.5cm}

      ОТЧЕТ\\
      по лабораторной работе
    }

    \vspace{4cm}
    {
      Преподаватель: \hfill Клебанов Борис Исаевич
    }
    \par
    {
      Студент: \hfill Сухоплюев Илья Владимирович
    }
    \par
    {
      Группа: \hfill РИ-440001
    }

    \par
    \vspace{3.5cm}
    Екатеринбург\\
    2017
\end{center}


\frontmatter % выключает нумерацию ВСЕГО; здесь начинаются ненумерованные главы: реферат, введение, глоссарий, сокращения и прочее.

% Команды \breakingbeforechapters и \nonbreakingbeforechapters
% управляют разрывом страницы перед главами.
% По-умолчанию страница разрывается.

% \nobreakingbeforechapters
% \breakingbeforechapters

% \include{00-abstract}
\pagestyle{plain}

\tableofcontents

% \include{10-defines}
% \include{11-abbrev}

\Introduction

Big Data и Data Maining становятся все более востребованными в бизнес-сфере.
Получение полезной информации из тон данных может помочь выявить
новые закономерности и использовать их. Поэтому очень полезно быть
знакомым с инструментами по аналитке данных. Один из таких инструментов --
Qlik.

Целью даннойлабораторной работы является ознакомление с программой QlikView,
на примере анализа Excel-таблиц о заказах в гипотетическом магазине.

\mainmatter % это включает нумерацию глав и секций в документе ниже

\chapter{Ход лабораторной работы}

\myImage{Выбираем "Новый документ", чтобы приступить к работе.}{1}{1}
\myImage{Мастер помогает нам выполнить первоначальную конфигурацию, выбирем данные, которые хотим анализировать.}{2}{2}
\myImage{После импорта данных программа предлагает нам подредактировать название колонок данных.}{3}{3}
\myImage{Сохраним полученный документ. так как мы пользуемся бесплатной лицензией он будет привязан к текущему компьютеру.}{4}{4}
\myImage{Менеджер создания графиков. ПОстроим столбчатю диаграмму.}{5}{5}
\myImage{Выбирем измирения, по которым хотим посмотреть диаграмму. Посмотрим отношение Покупатель/количество товара.}{6}{6}
\myImage{Выберием представление данных, с которым нам будет удобно работать}{7}{7}
\myImage{Построим также отношение Товары/Количество. На графиках наглядно видно, что PageWeavers
приобрел больше всего товаров, при этом основная масса продаж является поштучечной (товар в одном экземпляре, купили более 400 раз.)}{8}{8}

\chapter{Использование редактора скрипта при создании приложений QlickView}

Цель работы: приобретение навыков создания приложений на основе платформы QlickView
с использованием редактора скрипта.

Задача работы (Вариант 2): Работа с базой данных «Деятельность музея».
Оценить загруженность сотрудников по обслуживанию на основе анализа общей
площади обслуживаемых залов для каждого сотрудника.

\myImage{С помощью кнопки «Новый» откройтем первую основную вкладку приложения(
Если открылся мастер начала работы, то закройтем его)}{11}{11}
\myImage{Выберитем в меню Файл команду «Редактор Скрипта»}{12}{12}

\myImage{Нажмем на кнопку «Соед» для выбора способа подключения к БД}{13}{13}
\myImage{Выберем интересующую нас БД}{14}{14}
\myImage{После чего в скрипт добавится строка-подключения к выбранной БД}{15}{15}
\myImage{С помощью конпки "Выбрать", добавим необходимые таблицы для нашего анализа}{16}{16}
\myImage{После выбора в редакторе запроса появятся соответствующие запросы по загрузке данных.}{17}{17}
\myImage{Сохраним наш документ. После чего загрузим (Ctrl+R) данные из БД.
Выберем необходимые столбцы в свойствах листа, которые должны быть отображены}{18}{18}

\myImage{Выбирая ответственного(3), видим что он следит за 213 метрами в квадрате}{19}{19}
\myImage{Смотря на сотрудника с темже идендификатором(3), мы видим, что у работника нет рассписания}{20}{20}
\myImage{Выбирая экскурсовода, видим что он не ответственнен за зал. Но имеет рабочий график}{21}{21}

\myImage{Построим Столбчатую, Круговую диаграмму, Гистаграму, а также сводную таблицу по
отношению Ответственный/площадь помещений, чтобы наблюдать загруженность сотрудников}{22}{22}

\myImage{Составим прямую таблицу, которая позволяет сортировать наблюдаемые данные}{23}{23}
\clearpage

Из полученных выводов, можем заметить, что ответственный(3) является сейчас одним из самых
загруженных (в плане площади помещения), что может быть поводом к перерапределению
нагрузки и/или денежных поощрений.


% \include{20-analysis}
% \include{30-design}
% \include{40-impl}
% \include{50-research}

\backmatter %% Здесь заканчивается нумерованная часть документа и начинаются ссылки и
            %% заключение

\Conclusion % заключение к отчёту

В результате проделанной лабораторной работы было проведено ознакомление
с программой QLikView, а также проведен анализ учебных данных взятых из
Excel-таблиц.

%%% Local Variables:
%%% mode: latex
%%% TeX-master: "rpz"
%%% End:


\include{81-biblio}

% \appendix   % Тут идут приложения

% \include{90-appendix1}
% \include{91-appendix2}

\end{document}

%%% Local Variables:
%%% mode: latex
%%% TeX-master: t
%%% End:
