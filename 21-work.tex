\chapter{Использование редактора скрипта при создании приложений QlickView}

Цель работы: приобретение навыков создания приложений на основе платформы QlickView
с использованием редактора скрипта.

Задача работы (Вариант 2): Работа с базой данных «Деятельность музея».
Оценить загруженность сотрудников по обслуживанию на основе анализа общей
площади обслуживаемых залов для каждого сотрудника.

\myImage{С помощью кнопки «Новый» откройтем первую основную вкладку приложения(
Если открылся мастер начала работы, то закройтем его)}{11}{11}
\myImage{Выберитем в меню Файл команду «Редактор Скрипта»}{12}{12}

\myImage{Нажмем на кнопку «Соед» для выбора способа подключения к БД}{13}{13}
\myImage{Выберем интересующую нас БД}{14}{14}
\myImage{После чего в скрипт добавится строка-подключения к выбранной БД}{15}{15}
\myImage{С помощью конпки "Выбрать", добавим необходимые таблицы для нашего анализа}{16}{16}
\myImage{После выбора в редакторе запроса появятся соответствующие запросы по загрузке данных.}{17}{17}
\myImage{Сохраним наш документ. После чего загрузим (Ctrl+R) данные из БД.
Выберем необходимые столбцы в свойствах листа, которые должны быть отображены}{18}{18}

\myImage{Выбирая ответственного(3), видим что он следит за 213 метрами в квадрате}{19}{19}
\myImage{Смотря на сотрудника с темже идендификатором(3), мы видим, что у работника нет рассписания}{20}{20}
\myImage{Выбирая экскурсовода, видим что он не ответственнен за зал. Но имеет рабочий график}{21}{21}

\myImage{Построим Столбчатую, Круговую диаграмму, Гистаграму, а также сводную таблицу по
отношению Ответственный/площадь помещений, чтобы наблюдать загруженность сотрудников}{22}{22}

\myImage{Составим прямую таблицу, которая позволяет сортировать наблюдаемые данные}{23}{23}
\clearpage

Из полученных выводов, можем заметить, что ответственный(3) является сейчас одним из самых
загруженных (в плане площади помещения), что может быть поводом к перерапределению
нагрузки и/или денежных поощрений.
